\documentclass[12pt]{article}
\usepackage[utf8]{inputenc}
\usepackage[T1]{fontenc}
\usepackage{mathptmx} % Times New Roman font
\usepackage{amsmath, amssymb}
\usepackage{enumitem} 
\usepackage{geometry}
\geometry{a4paper, margin=1in}
\usepackage{graphicx}
\usepackage{titlesec}
\usepackage{lipsum} % For dummy text (temporary usage for the rudimentary stages of the project) 

% Custom title format
\titleformat{\section}{\normalfont\Large\bfseries}{\thesection}{1em}{}

% Title Page
\title{ 
    \vspace{2in} % Adds vertical space of 2 inches before the title text.
    \textbf{Take-Home Eksamen DM500 Efterår 2023}\\ % Sets the main title in bold font and breaks the line after it.
    \vspace{0.5in} % Adds additional vertical space of 0.5 inches after the main title.
    \large University of Southern Denmark\\ % Sets the following text in a large font size for the university name and breaks the line after it.
    \vspace{1.5in} % Finally, adds 1.5 inches of vertical space after the logo before ending the title block.
} 

\author{Jacques Klimin Buis, Sabirin Ismail Ali}

\date{\today}

\begin{document}

\maketitle
\thispagestyle{empty} % Removes page number from title page
\newpage

% Table of Contents
\tableofcontents
\thispagestyle{empty}
\newpage

% Start of Content
\setcounter{page}{1}

\section{Reeksamen februar 2015 opgave 2}

\subsection{Opgave}
\begin{enumerate}[label=\alph*)] % This will label the top-level items as a), b), etc.
    \item Hvilke af følgende udsagn er sande?
    \begin{enumerate}[label=\arabic*.] % This will label nested items as 1., 2., etc.
        \item \(\forall x \in \mathbb{N}: \exists y \in \mathbb{N}: x < y\)
        \item \(\forall x \in \mathbb{N}: \exists! y \in \mathbb{N}: x < y\)
        \item \(\exists y \in \mathbb{N}: \forall x \in \mathbb{N}: x < y\)
    \end{enumerate}
    \item Angiv negeringen af udsagn 1. fra spørgsmål a). Negerings-operatoren (\(\neg\)) må ikke indgå i dit udsagn.
\end{enumerate}

\subsection{Løsning}
\subsubsection{Løsning til a)}
Vi analyserer de givne udsagn i forhold til deres sandhedsværdier inden for mængden af naturlige tal \(\mathbb{N}\):

\vspace{0.25in}

\textbf{1.} Betragt ethvert element \(x \in \mathbb{N}\). Baseret på de naturlige tals fundamentale velordnede og uendelige egenskaber, eksisterer der ubestridt altid et element \(y \in \mathbb{N}\), således at \(y = x + 1\), hvilket umiddelbart medfører, at \(x < y\). Denne konklusion bekræfter, at for hvert element i \(\mathbb{N}\), findes der et større element, og bekræfter dermed udsagnets sandhed. 

\vspace{0.25in}

\textbf{2.} For ethvert element \(x \in \mathbb{N}\), er der ikke blot ét unikt større element \(y \in \mathbb{N}\), men derimod en uendelig mængde af sådanne elementer. Dette strider imod udsagnets præmis, hvori det hævdes, at der eksisterer præcis ét sådant element \(y\). Derfor er udsagnet falsk.

\vspace{0.25in}

\textbf{3.} Udsagnet postulerer eksistensen af et element \(y \in \mathbb{N}\), som er større end alle andre elementer i \(\mathbb{N}\). Denne påstand er umiddelbart modstridende med de naturlige tals fundamentale egenskab om at være en uendelig mængde. Ingen enkelt element i \(\mathbb{N}\) kan være større end alle andre, da der altid vil være et større element. Under hensyntagen til den nævnte betingelse kan man ikke erklære udsagnet for sandt.


\subsubsection{Løsning til b)}
Den symbolske form for negeringen af det logiske udtryk præsenteres nedenfor:
\[
\exists x \in \mathbb{N} : \forall y \in \mathbb{N}, x \geq y
\]
Negeringen af det første udsagn, uden at bruge negerings-operatoren, kan formuleres som: "Der eksisterer et naturligt tal \(x\), så for ethvert naturligt tal \(y\), gælder det at \(x \geq y\)." Dette udsagn er \textbf{falsk}, da der for ethvert valgt naturligt tal \(x\) altid vil findes et andet naturligt tal \(y\), der er større end \(x\).


\section{Reeksamen februar 2015 opgave 3}
\subsection{Opgave}
Lad \( R, S \) og \( T \) være binære relationer på mængden \( \{1,2,3,4\} \).

\begin{itemize}
  \item[a)] Lad \( R = \{(1,1), (2,1), (2,2), (2,4), (3,1), (3,3), (3,4), (4,1), (4,4)\} \). \\ Er \( R \) en partiel ordning?
  \item[b)] Lad \( S = \{(1,2), (2,3), (2,4), (4,2)\} \). \\ Angiv den transitive lukning af \( S \).
  \item[c)] Lad \( T = \{(1,1), (1,3), (2,2), (2,4), (3,1), (3,3), (4,2), (4,4)\} \). \\ Bemærk, at \( T \) er en ækvivalens-relation.\\ Angiv \( T \)'s ækvivalens-klasser.
\end{itemize}

\subsection{Løsning}
\subsubsection{Løsning til a)}
For at være en partiel ordning skal en relation \( R \) være refleksiv, antisymmetrisk og transitiv.
\begin{itemize}
  \item Refleksivitet: En relation er refleksiv, hvis ethvert element er relateret til sig selv. Da alle par \( (1,1), (2,2), (3,3), (4,4) \) er i \( R \), opfylder \( R \) kravet om refleksivitet.
  \item Antisymmetri: En relation er antisymmetrisk, hvis der for alle par \( (a,b) \) og \( (b,a) \) i \( R \), følger at \( a=b \). Relationen \( R \) indeholder ikke noget par \( (a,b) \) og \( (b,a) \) hvor \( a \neq b \), hvilket betyder at \( R \) er antisymmetrisk.
  \item Transitivitet: For at \( R \) er transitiv, skal det gælde, at hvis \( (a,b) \) og \( (b,c) \) er i \( R \), så skal \( (a,c) \) også være i \( R \). På grundlag af de givne par kan vi se, at relationen ikke indeholder et modstridende tilfælde, der ville bryde transitiviteten. Derfor er \( R \) transitiv.
\end{itemize}
Da \( R \) opfylder alle tre betingelser, kan vi konkludere, at \( R \) er en partiel ordning.

\subsubsection{Løsning til b)}
Den transitive lukning af en relation \( S \) er den mindste transitive relation, der indeholder \( S \). For at finde den transitive lukning, ser vi på alle par af elementer \( (a,b) \) og \( (b,c) \) i \( S \) og tilføjer \( (a,c) \) til \( S \), hvis det ikke allerede er der. For \( S \) bliver den transitive lukning \( S \cup \{(1,3), (1,4), (2, 2), (4, 3), (4, 4)\} \).

\subsubsection{Løsning til c)}
For at bestemme ækvivalensklasserne for relationen \( T \), skal vi identificere alle elementer der er relateret til hinanden. En ækvivalensklasse for et element \( a \) er mængden af alle elementer der er relateret til \( a \) inklusive \( a \) selv. For relationen \( T \) kan vi opstille følgende ækvivalensklasser:

\begin{itemize}
    \item Ækvivalensklassen for 1: [1] = \{1, 3\}
    \item Ækvivalensklassen for 2: [2] = \{2, 4\}
    \item Ækvivalensklassen for 3: [3] = \{1, 3\} (Bemærk at dette er den samme klasse som for 1, da 1 og 3 er relateret til hinanden)
    \item Ækvivalensklassen for 4: [4] = \{2, 4\} (Ligesom med 3, er denne klasse identisk med klassen for 2)
\end{itemize}

Da ækvivalensklasserne er disjunkte og hver klasse repræsenterer en partionering af den oprindelige mængde, er det klart at ækvivalensklasserne for \( T \) er \{[1], [2]\}, hvilket også kan skrives som \{\{1, 3\}, \{2, 4\}\}. Disse klasser dækker hele mængden \( \{1, 2, 3, 4\} \), hvilket bekræfter at \( T \) er en ækvivalensrelation.


\section{Matricer for Relationer R, S og T}
Relationernes matricer repræsenterer tilstedeværelsen af par i hver relation som binære værdier i en matrix form. En '1' i en celle repræsenterer at et par (række, kolonne) er i relationen, og en '0' betyder at paret ikke er i relationen. Her er de tilsvarende matricer:

\subsection{Matrix for Relation R}
Relation \( R \) er en partiel ordning. Dens matrix repræsentation er:
\[
\begin{pmatrix}
1 & 0 & 0 & 0 \\
1 & 1 & 0 & 1 \\
1 & 0 & 1 & 1 \\
1 & 0 & 0 & 1 \\
\end{pmatrix}
\]
Hver celle \( (i, j) \) i matrixen er '1' hvis \( (a_i, a_j) \) er i \( R \), ellers '0'.

\subsection{Matrix for Relation S og dens transitive lukning}
For relation \( S \) og dens transitive lukning, har vi:
\[
\text{Original S:}
\begin{pmatrix}
0 & 1 & 0 & 0 \\
0 & 0 & 1 & 1 \\
0 & 0 & 0 & 0 \\
0 & 1 & 0 & 0 \\
\end{pmatrix}
\quad
\text{Transitiv lukning af S:}
\begin{pmatrix}
0 & 1 & 1 & 1 \\
0 & 1 & 1 & 1 \\
0 & 0 & 0 & 0 \\
0 & 1 & 1 & 1 \\
\end{pmatrix}
\]
Den transitive lukning inkluderer de originale par samt de par der fuldender transitiviteten.

\subsection{Matrix for Ækvivalensrelation T}
Da \( T \) er en ækvivalensrelation, vil dens matrix være symmetrisk om diagonalen og have '1'-ere på diagonalen:
\[
\begin{pmatrix}
1 & 0 & 1 & 0 \\
0 & 1 & 0 & 1 \\
1 & 0 & 1 & 0 \\
0 & 1 & 0 & 1 \\
\end{pmatrix}
\]
Ækvivalensklasserne \{1, 3\} og \{2, 4\} fremgår klart af de '1'-ere, som ikke er på diagonalen.

Disse matricer hjælper med at visualisere egenskaberne af de binære relationer og kan bruges til at afsløre yderligere egenskaber og relationer mellem elementerne i de givne sæt.

\section{Reeksamen 13. Januar 2012 opgave 1}
\subsection{Opgave}
Betragt funktionerne f∶ R → R og g∶ R → R defineret ved

\vspace{0.25in}
 
\begin{math}f(x) = x^2 + x + 1 \end{math} og
\begin{math}g(x) = 2x - 2 \end{math}

\begin{enumerate}[label=\alph*)] % This will label the top-level items as a), b), etc.
\item[a)] Er f en bijektion?
\item[b)] Har f en invers funktion? 
\item[c)] Angiv f + g. 
\item[d)] Angiv g∘f.
\end{enumerate}



\subsection{Løsning}
\subsubsection{Løsning til a)}

For at bevise at funktionen \( f \) er injektiv, antager vi at \( f(x_1) = f(x_2) \) for to vilkårlige \( x_1 \) og \( x_2 \) i definitionsmængden.

\[ x_1^2 + x_1 + 1 = x_2^2 + x_2 + 1 \]

Subtraher \( 1 \) fra begge sider:

\[ x_1^2 + x_1 = x_2^2 + x_2 \]

Subtraher \( x_2^2 + x_2 \) fra begge sider for at isolere \( x_1^2 - x_2^2 \) på den ene side og \( x_1 - x_2 \) på den anden:

\[ x_1^2 - x_2^2 + x_1 - x_2 = 0 \]

Faktorisér venstre side ved at anvende forskellen på to kvadrater og faktorisering:

\[ (x_1 - x_2)(x_1 + x_2) + (x_1 - x_2) = 0 \]

Træk \( x_1 - x_2 \) ud som en fælles faktor:

\[ (x_1 - x_2)(x_1 + x_2 + 1) = 0 \]

For at produktet skal være nul, skal mindst en af faktorerne være nul. Da \( x_1 + x_2 + 1 \) ikke kan være nul for alle reelle tal \( x_1 \) og \( x_2 \) (da det altid vil være større end 1), må vi have at \( x_1 - x_2 = 0 \), hvilket giver:

\[ x_1 = x_2 \]

Dermed er det bevist at \( f \) er injektiv.

Nu tjekker vi om f er surjektivt, for at funktionen kan være surjektivt, skal x værdien mindst have en korresponderende y værdi. Dette tjekkes ved at kigge på funktionens graf. Der f er en andengradsligning kan vi se på dens graf, at det er et polynomium. Det vil sige at, grafen er positiv, der og grenene vender opad. Vi kan nu kigge på diskriminanten for at regne, dens minimum ekstremum ud. Vi bruger følgende formel \begin{math} d=b^2-4ac\end{math}

\begin{math}d=1^2-4·1·1=-3\end{math}

Vi kan nu se at diskriminanten er et negativt tal -3, dette vil medføre til, at f ikke skærer x. aksen og dermed kan den ikke have negative værdier. Det vil sige, at y ikke kan være mindre end 1, og der vil derfor ikke være nogen x værdier som kan opnå dette. Vi kan nu konkludere at funktion ikke er surjektiv, der den ikke kan få negative værdier, men kun positive. Det betyder at den ikke dækker, både negative og positive tal. Så kan funktion ikke have en y værdi som mindst har en tilhørende x værdi. 

\subsubsection{Løsning til b)}

Den inverse funktion til f, skrives med udtrykket \begin{math} f^-1 \end{math}, som betyder det omvendte, der f ikke er bijektivt, så har den ikke en invers funktion. 




\subsubsection{Løsning til c)}
Angiv \begin{math}f + g \end{math}

\begin{math} f+g=(x^2 + x + 1) +(2x - 2) \end{math}

\text{Så reducer man bare.}

\begin{math} f+g=x^2+3x-1 \end{math}

\subsubsection{Løsning til d)}

Angiv\begin{document} \( g \circ f \)  


\text{g er den ydre funktion og f er den indre. Det vil sige at vi sætter f(x) ind på x's plads i g(x)}


\begin{math} g(f(x))=2(x^2+x+1)-2 \end{math}

\text{Nu ganger man 2 ind i parentesen. }
 
\begin{math}g(f(x))=2x^2+2x+2-2\end{math}

\text{2 går ud med 2, hvor så vi har følgende udtryk tilbage:}

\begin{math}g(f(x))=2x^2+2x\end{math}

\end{document}
